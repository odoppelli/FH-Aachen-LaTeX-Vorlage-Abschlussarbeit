\clearpage
\chapter{\textbf{Einleitung}}\label{chap:Einleitung}
\addtocontents{toc}{\vspace{0.8cm}}

% TODO Einleitung schreiben
Roboter werden inzwischen sowohl in der Industrie, als teilweise auch schon in privateren Bereichen vermehrt eingesetzt. Im industriellen Umfeld sind das oft stationäre Roboter, die sich wiederholende Aktionen mit starker Präzision, Zuverlässigkeit, Geschwindigkeit und Ausdauer ausführen, oder aber bodengebundenen mobile Roboter, die Waren und Güter vom Ort A nach Ort B transportieren. Im privaten Gebrauch wird vermehrt von mobilen Robotern im Haushalt Gebrauch gemacht. Beispielsweise werden Staubsaugerroboter mit der Aufgabe eines sauberen Bodens vertraut gemacht, Rasenmäherroboter mit der Instandhaltung des eigenen Rasens betraut oder Multicopter zum Filmen von Landschaften genutzt.\\
Letzteres System erhält nun auch mehr und mehr Einzug in die Industrie, gepaart mit einem starken Gedanken an die Autonomie dieser Roboter. Sogenannte \textit{unmanned aerial vehicle} (kurz UAV) (deutsch: \textit{unbemannte-Luft-Vehikel}) sollen eigenständig in geschlossenen Räumen zum schnellen Transport von leichteren Gegenständen genutzt werden, oder outdoor eingesetzt werden um beispielsweise die Felder eines Bauern zu überwachen, die Feuerwehr beim Löschen eines Brandes aus der Luft unterstützen oder aber eine Landschaft zu kartographieren.\\
Hier soll die Herausforderung behandelt werden ein UAV so mit Sensorik und Algorithmen auszustatten, dass es sich outdoor möglichst genau lokalisieren kann, mit dem späteren Ziel der Lokalisierung anderer Objekte. In dieser Arbeit wird ersteres Problem, das der Lokalisierung des UAVs, beleuchtet. Dazu werden in Kapitel \ref{chap:Vorwissen} grundsätzliche Begrifflichkeiten und Definitionen vorgestellt, damit die darauf aufbauenden Ansätze zur Lösung des Lokalisierungsproblems mobiler Roboter nachvollziehbar werden. Die dabei vorgeschlagenen Vorgehensweisen zur Lokalisierung werden dabei strukturell aufgearbeitet und in die genauen Ziele, die Rahmenbedingungen und gegebenenfalls auch die Testergebnisse eingeteilt. 

%\newpage
