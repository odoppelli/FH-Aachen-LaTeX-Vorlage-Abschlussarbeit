\chapter{\textbf{Fazit}}\label{chap:Fazit}
\addtocontents{toc}{\vspace{0.8cm}}

Für das vorhandene Problem der Lokalisierung eines UAV im urbanen Raum wäre eine Kombination aus den in Kapitel \ref{chap:Lokalisierung - Lösungsansätze} erläuterten Algorithmen denkbar. Der Leader-Based Bat Algorithm \ref{sec:lbba} ist dazu in der Lage eine genaue Positionsbestimmung mit nur einer Messung aller externen Sensoren und einer gegebenen Karte, beziehungsweise einem Modell der Umwelt, durchzuführen. Im Vergleich zur Monte Carlo Lokalisierung \ref{sec:mcl} und der Adaptive Monte Carlo Lokalisierung braucht der LBBA allerdings mehr Rechenpower und -zeit wenn es um das Position Tracking geht.
Das in  Sektion \ref{sec:vehicle_urban} erläuterte System hat gezeigt, dass AMCL dazu in der Lage ist eine genaue Position in Echtzeit zu berechnen, um damit ein Golf-Cart autonom um einen Campus fahren zu lassen.\\
Im Hinblick auf die Herausforderung der 3D Lokalisierung des UAV in einer komplexeren Umgebung, könnte ein GPS genutzt werden um eine ziemlich ungenaue, aber doch ausreichend informative, Aussage über die Position des Roboters treffen zu können. Dann könnten in dem durch das GPS bereits eingegrenzten Raum eine bestimmte Anzahl an Partikeln initiiert werden. Um mit diesen eine globale Positionsbestimmung durchzuführen könnte entweder direkt AMCL angewendet werden, oder zuerst LBBA eingesetzt werden, um bei gefundenem Standort diese bereits konvergierten Partikel für einen AMCL Algorithmus zu verwenden, und so das Position Tracking durchzuführen.